% 重要: LuaLaTeX エンジンでコンパイルしてください
\documentclass{ltjsarticle}
%----------------Load packages---------------------

\usepackage{amsmath,amsthm,amssymb}
% \usepackage{unicode-math}
% \setmathfont{Latin Modern Math}

\usepackage{graphicx, color}

\usepackage[style=alphabetic,backend=biber]{biblatex}
\addbibresource{ref.bib}  % .bib ファイルを指定

\renewcommand{\emph}{\textbf}
%----------------Define theorem environments---------------------

\theoremstyle{definition}
\newtheorem{prop}{Proposition}[section]
\newtheorem*{prop*}{Proposition}
\newtheorem{thm}[prop]{Theorem}
\newtheorem*{thm*}{Theorem}
\newtheorem{cor}[prop]{Corollary}
\newtheorem{dfn}[prop]{Definition}
\newtheorem{exm}[prop]{Example}
\newtheorem{rem}[prop]{Remark}

%-----------------------------------------------

\begin{document}

%----------------Title and Author---------------------
\title{ドクワキ!ピタゴラの定理}
\author{小松, 藤吉, 薄井}
\date{\today}
\maketitle

\tableofcontents

%----------------Abstract---------------------
\begin{abstract}
    ピタゴラの定理を紹介します.
\end{abstract}

%----------------Main text---------------------
\section{はじめに}
ピタゴラの定理は, 直角三角形の辺の長さの関係を表す有名な定理です.
本稿では, ピタゴラの定理を紹介し, その証明を示します.

\section{ピタゴラの定理}
\begin{thm}[ピタゴラの定理]
    直角三角形において, 斜辺の長さの二乗は, 他の二辺の長さの二乗の和に等しい.
    すなわち, 直角三角形の辺の長さをそれぞれ $a$, $b$, $c$ とするとき, 以下の関係が成り立つ:
    \[
        c^2 = a^2 + b^2
    \]
\end{thm}
\begin{proof}
    直角三角形を並べて一辺の長さが$a+b$の正方形を作る. この正方形の面積を$S$とすると,
    \[
        S = (a + b)^2
    \]
    である.
    一方で, この正方形は四つの直角三角形と一辺の長さが$c$の正方形に分割できるので,
    \[
        S = c^2 + 2ab
    \]
    である.
    この二つを比較して, 整理すると
    \[
        c^2 = a^2 + b^2 
    \]
\end{proof}

\section{Euclid幾何学との関係}
ピタゴラスの定理は Euclid 幾何学における距離の定義に直接関連している.
\(\mathbb{R}\) を実数全体とし, 集合 \(\mathbb{R}^2\) に次のように距離 \(d\colon \mathbb{R}^2 \times \mathbb{R}^2 \to \mathbb{R}\) を定義する:
\[
    d\bigl((x_1, y_1), (x_2, y_2)\bigr) = \sqrt{(x_2 - x_1)^2 + (y_2 - y_1)^2}.
\]
この距離は Euclid 幾何学の公理を満たし, 特に三角不等式を満たすことが知られている.
すなわち, 任意の点 \(A = (x_1, y_1)\), \(B = (x_2, y_2)\), \(C = (x_3, y_3)\) に対して,
\[
    d(A, C) \leq d(A, B) + d(B, C)
\]
が成り立つ.
ここで等号が成立するのは, 点 \(A\), \(B\), \(C\) が一直線上にある場合である.
ピタゴラスの定理は, 直角三角形における辺の長さの関係を示すものであり, この距離の定義に基づいて証明されていた.

この距離を \(\mathbb{R}^n\) 上に拡張することができる.
すなわち, 任意の点 \(P = (x_1, x_2, \ldots, x_n)\), \(Q = (y_1, y_2, \ldots, y_n)\) に対して,
\[
    d(P, Q) = \sqrt{(y_1 - x_1)^2 + (y_2 - x_2)^2 + \cdots + (y_n - x_n)^2}
\]
と定義する.
この距離も Euclid 幾何学の公理を満たし, ピタゴラスの定理は高次元空間においても成立する.
このように, ピタゴラスの定理は Euclid 幾何学における距離の定義と密接に関連しており, 幾何学的な性質を理解する上で重要な役割を果たしている.

このように, ピタゴラスの定理から導かれる距離を用いた位相を備えた空間 \(\mathbb{R}^n, d\) は\emph{Euclid空間}と呼ばれ, 現代数学の多くの概念はこのEuclid空間を基礎として構築されていった.

\end{document}