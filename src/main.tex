% 重要: LuaLaTeX エンジンでコンパイルしてください
\documentclass{ltjsarticle}
%----------------Load packages---------------------

\usepackage{amsmath,amsthm,amssymb}
% \usepackage{unicode-math}
% \setmathfont{Latin Modern Math}

\usepackage{graphicx, color}

\usepackage[style=alphabetic,backend=biber]{biblatex}
\addbibresource{ref.bib}  % .bib ファイルを指定

\renewcommand{\emph}{\textbf}
%----------------Define theorem environments---------------------

\theoremstyle{definition}
\newtheorem{prop}{Proposition}[section]
\newtheorem*{prop*}{Proposition}
\newtheorem{thm}[prop]{Theorem}
\newtheorem*{thm*}{Theorem}
\newtheorem{cor}[prop]{Corollary}
\newtheorem{dfn}[prop]{Definition}
\newtheorem{exm}[prop]{Example}
\newtheorem{rem}[prop]{Remark}

%-----------------------------------------------

\begin{document}

%----------------Title and Author---------------------
\title{ドクワキ!ピタゴラの定理}
\author{小松, 藤吉, 薄井}
\date{\today}
\maketitle

\tableofcontents

%----------------Abstract---------------------
\begin{abstract}
    ピタゴラの定理を紹介します.
\end{abstract}

%----------------Main text---------------------
\section{はじめに}
ピタゴラの定理は, 直角三角形の辺の長さの関係を表す有名な定理です.
本稿では, ピタゴラの定理を紹介し, その証明を示します.

\section{ピタゴラの定理}
\begin{thm}[ピタゴラの定理]
    直角三角形において, 斜辺の長さの二乗は, 他の二辺の長さの二乗の和に等しい.
    すなわち, 直角三角形の辺の長さをそれぞれ $a$, $b$, $c$ とするとき, 以下の関係が成り立つ:
    \[
        c^2 = a^2 + b^2
    \]
\end{thm}

\end{document}