ピタゴラスの定理は Euclid 幾何学における距離の定義に直接関連している.
\(\mathbb{R}\) を実数全体とし, 集合 \(\mathbb{R}^2\) に次のように距離 \(d\colon \mathbb{R}^2 \times \mathbb{R}^2 \to \mathbb{R}\) を定義する:
\[
    d\bigl((x_1, y_1), (x_2, y_2)\bigr) = \sqrt{(x_2 - x_1)^2 + (y_2 - y_1)^2}.
\]
この距離は Euclid 幾何学の公理を満たし, 特に三角不等式を満たすことが知られている.
すなわち, 任意の点 \(A = (x_1, y_1)\), \(B = (x_2, y_2)\), \(C = (x_3, y_3)\) に対して,
\[
    d(A, C) \leq d(A, B) + d(B, C)
\]
が成り立つ.
ここで等号が成立するのは, 点 \(A\), \(B\), \(C\) が一直線上にある場合である.
ピタゴラスの定理は, 直角三角形における辺の長さの関係を示すものであり, この距離の定義に基づいて証明されていた.

この距離を \(\mathbb{R}^n\) 上に拡張することができる.
すなわち, 任意の点 \(P = (x_1, x_2, \ldots, x_n)\), \(Q = (y_1, y_2, \ldots, y_n)\) に対して,
\[
    d(P, Q) = \sqrt{(y_1 - x_1)^2 + (y_2 - x_2)^2 + \cdots + (y_n - x_n)^2}
\]
と定義する.
この距離も Euclid 幾何学の公理を満たし, ピタゴラスの定理は高次元空間においても成立する.
このように, ピタゴラスの定理は Euclid 幾何学における距離の定義と密接に関連しており, 幾何学的な性質を理解する上で重要な役割を果たしている.

このように, ピタゴラスの定理から導かれる距離を用いた位相を備えた空間 \(\mathbb{R}^n, d\) は\emph{Euclid空間}と呼ばれ, 現代数学の多くの概念はこのEuclid空間を基礎として構築されていった.